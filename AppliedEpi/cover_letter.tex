% Options for packages loaded elsewhere
\PassOptionsToPackage{unicode}{hyperref}
\PassOptionsToPackage{hyphens}{url}
\PassOptionsToPackage{dvipsnames,svgnames,x11names}{xcolor}
%
\documentclass[
  letterpaper,
  DIV=11,
  numbers=noendperiod]{scrartcl}

\usepackage{amsmath,amssymb}
\usepackage{lmodern}
\usepackage{iftex}
\ifPDFTeX
  \usepackage[T1]{fontenc}
  \usepackage[utf8]{inputenc}
  \usepackage{textcomp} % provide euro and other symbols
\else % if luatex or xetex
  \usepackage{unicode-math}
  \defaultfontfeatures{Scale=MatchLowercase}
  \defaultfontfeatures[\rmfamily]{Ligatures=TeX,Scale=1}
\fi
% Use upquote if available, for straight quotes in verbatim environments
\IfFileExists{upquote.sty}{\usepackage{upquote}}{}
\IfFileExists{microtype.sty}{% use microtype if available
  \usepackage[]{microtype}
  \UseMicrotypeSet[protrusion]{basicmath} % disable protrusion for tt fonts
}{}
\makeatletter
\@ifundefined{KOMAClassName}{% if non-KOMA class
  \IfFileExists{parskip.sty}{%
    \usepackage{parskip}
  }{% else
    \setlength{\parindent}{0pt}
    \setlength{\parskip}{6pt plus 2pt minus 1pt}}
}{% if KOMA class
  \KOMAoptions{parskip=half}}
\makeatother
\usepackage{xcolor}
\setlength{\emergencystretch}{3em} % prevent overfull lines
\setcounter{secnumdepth}{-\maxdimen} % remove section numbering
% Make \paragraph and \subparagraph free-standing
\ifx\paragraph\undefined\else
  \let\oldparagraph\paragraph
  \renewcommand{\paragraph}[1]{\oldparagraph{#1}\mbox{}}
\fi
\ifx\subparagraph\undefined\else
  \let\oldsubparagraph\subparagraph
  \renewcommand{\subparagraph}[1]{\oldsubparagraph{#1}\mbox{}}
\fi


\providecommand{\tightlist}{%
  \setlength{\itemsep}{0pt}\setlength{\parskip}{0pt}}\usepackage{longtable,booktabs,array}
\usepackage{calc} % for calculating minipage widths
% Correct order of tables after \paragraph or \subparagraph
\usepackage{etoolbox}
\makeatletter
\patchcmd\longtable{\par}{\if@noskipsec\mbox{}\fi\par}{}{}
\makeatother
% Allow footnotes in longtable head/foot
\IfFileExists{footnotehyper.sty}{\usepackage{footnotehyper}}{\usepackage{footnote}}
\makesavenoteenv{longtable}
\usepackage{graphicx}
\makeatletter
\def\maxwidth{\ifdim\Gin@nat@width>\linewidth\linewidth\else\Gin@nat@width\fi}
\def\maxheight{\ifdim\Gin@nat@height>\textheight\textheight\else\Gin@nat@height\fi}
\makeatother
% Scale images if necessary, so that they will not overflow the page
% margins by default, and it is still possible to overwrite the defaults
% using explicit options in \includegraphics[width, height, ...]{}
\setkeys{Gin}{width=\maxwidth,height=\maxheight,keepaspectratio}
% Set default figure placement to htbp
\makeatletter
\def\fps@figure{htbp}
\makeatother

\KOMAoption{captions}{tableheading}
\makeatletter
\makeatother
\makeatletter
\makeatother
\makeatletter
\@ifpackageloaded{caption}{}{\usepackage{caption}}
\AtBeginDocument{%
\ifdefined\contentsname
  \renewcommand*\contentsname{Table of contents}
\else
  \newcommand\contentsname{Table of contents}
\fi
\ifdefined\listfigurename
  \renewcommand*\listfigurename{List of Figures}
\else
  \newcommand\listfigurename{List of Figures}
\fi
\ifdefined\listtablename
  \renewcommand*\listtablename{List of Tables}
\else
  \newcommand\listtablename{List of Tables}
\fi
\ifdefined\figurename
  \renewcommand*\figurename{Figure}
\else
  \newcommand\figurename{Figure}
\fi
\ifdefined\tablename
  \renewcommand*\tablename{Table}
\else
  \newcommand\tablename{Table}
\fi
}
\@ifpackageloaded{float}{}{\usepackage{float}}
\floatstyle{ruled}
\@ifundefined{c@chapter}{\newfloat{codelisting}{h}{lop}}{\newfloat{codelisting}{h}{lop}[chapter]}
\floatname{codelisting}{Listing}
\newcommand*\listoflistings{\listof{codelisting}{List of Listings}}
\makeatother
\makeatletter
\@ifpackageloaded{caption}{}{\usepackage{caption}}
\@ifpackageloaded{subcaption}{}{\usepackage{subcaption}}
\makeatother
\makeatletter
\@ifpackageloaded{tcolorbox}{}{\usepackage[many]{tcolorbox}}
\makeatother
\makeatletter
\@ifundefined{shadecolor}{\definecolor{shadecolor}{rgb}{.97, .97, .97}}
\makeatother
\makeatletter
\makeatother
\ifLuaTeX
  \usepackage{selnolig}  % disable illegal ligatures
\fi
\IfFileExists{bookmark.sty}{\usepackage{bookmark}}{\usepackage{hyperref}}
\IfFileExists{xurl.sty}{\usepackage{xurl}}{} % add URL line breaks if available
\urlstyle{same} % disable monospaced font for URLs
\hypersetup{
  pdftitle={Cover letter, Applied Epi},
  pdfauthor={M. Rolland},
  colorlinks=true,
  linkcolor={blue},
  filecolor={Maroon},
  citecolor={Blue},
  urlcolor={Blue},
  pdfcreator={LaTeX via pandoc}}

\title{Cover letter, Applied Epi}
\author{M. Rolland}
\date{24/11/2024}

\begin{document}
\maketitle
\ifdefined\Shaded\renewenvironment{Shaded}{\begin{tcolorbox}[frame hidden, borderline west={3pt}{0pt}{shadecolor}, interior hidden, sharp corners, enhanced, breakable, boxrule=0pt]}{\end{tcolorbox}}\fi

To whom it may concern,

Here is my cover letter to be part of Applied Epi's roster of part-time
R instructors.

I have a strong theoretical background in biostatistics and
epidemiology. I have a masters degree in biostatistics (2009), which I
followed with a 2 year training to field epidemiology (the French FETP
program, PROFET, 2009-2011, which has since then merged with EPIET) and
I will be defending my thesis in statistics applied to epidemiology in
2023.

For the past 14 years I have worked in the field of statistics and data
science applied to epidemiology, medical research and public health,
both in France and in the USA (2012-2016).

I am a passionate R user, strong advocate of the tidyverse and I
strongly believe in the promotion of free open source tools for science
in general and public health in particular. I also am a great supporter
of the Epi R Handbook initiative and share it whenever I have the
opportunity.

I have taught R to my colleagues throughout my career, e.g.~when I was
working in Bordeaux' public health school (Isped) I taught R and the
tidyverse to many members of the Alima NGO with whom I was sharing
offices (\texttt{\{ggplot2\}}, \texttt{\{dplyr\}}, \texttt{\{tidyr\}},
etc). I have given many R related talks with the R user group of my city
(Grenoble, FR) of which I have been the coordinator since 2019 (website
and links to YouTube channel here :
(\url{r-in-grenoble.github.io/sessions.html}). I have also taught
several applied statistics classes for the Grenoble University, mainly
linear and logistic regression, applied to epidemiology for healthcare
practitioners : nurses, doctors, etc.

I know many of the R packages used for applied epidemiology, I am the
co-author of the Epidemiology CRAN task view
(\href{cran.r-project.org/web/views/Epidemiology.html}{cran.r-project.org/web/views/Epidemiology.html)})
and was a project manager for the R Epidemics Consortium (RECON) to
develop their task manager (\url{tasks.repidemicsconsortium.org/}).

I hope this experience illustrates that I have the knowledge and
experience to undertake your « general » R/epi modules and also many of
the more specific ones. Among the specific skills that are mentioned, I
have strong experience in data visualisation with \texttt{\{ggplot2\}},
reporting with \texttt{\{rmarkdown\}} and now \texttt{\{quarto\}},
working with Git and GitHub/GitLab, working with geospatial data and I
use RedCap weekly. However I don't have much experience with using
\texttt{\{shiny\}} as I have never had a project that I needed it for,
and have never used SQL with R.

I speak French (mother tongue) and English (my mother is British) and a
little German. I am willing to travel internationally. I live in
Grenoble (FR). Regarding availability, I am open to discussion but for
2023 I will probably be available for one or two week-long courses plus
a couple of half day courses each month. I am comfortable with teaching
at a distance but I prefer working in-person, especially for teaching.

I am a sociable and patient person and enjoy teaching, R and field
epidemiology, and truly believe my application is worth considering.

I am available for any further question and hope to hear from you soon.

Kind regards,\\
Matthieu



\end{document}
